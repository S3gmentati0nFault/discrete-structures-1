\subsection*{Exercise 2}
\boldmath
\textbf{Prove or disprove: A connected graph $G$ is bipartite if and only if for any $uv \in E(G)$ there is no $w \in V(G)$ with $dist_G(u,w) = dist_G(v,w).$} \\
\unboldmath 
\linebreak 
As $G$ is bipartite there exists a bipartition $(A, B)$ s.t. $V(A) \cap V(B) = \emptyset$ and $V(A) + V(B) = V(G).$  Assume $u \in V(A)$ and $v \in V(B)$. The claim asserts that there cannot exist a vertex $w$ in either $V(A)$ or $V(B)$ s.t. $dist_G(u, w) = dist_G(v, w)$. \\
\linebreak 
\boldmath
\textbf{$(\Rightarrow)$ If for any $uv \in E(G)$ there is no $w$ such that $dist_G(u,w) = dist_G(v,w)$, then $G$ is bipartite.} \\
\unboldmath
\linebreak 
Without loss of generality, we can assume for a contradiction, that $\exists$ such a vertex $w \in V(A)$. This means $\exists$ a path $uPw$ and a path $vPw$ of equal length. As $G$ is connected both paths must exist and so have length 
$>$ 1. \\
\linebreak
However, as $u$ and $w$ are $\in A$, any path between them must have an equal length. As $v$ and $w$ are in separate components of the bipartition, any path between them must have odd length. \\
\linebreak 
An even number can never equal an odd number and vice versa, and so $dist_G(u, w)$ can never equal $dist_G(v, w)$ for any $w$. Therefore, no such $w$ can exist.\\
\linebreak 
From this contradiction we know that a connected graph $G$ is bipartite if for any $uv \in E(G)$ there is no $w \in V(G)$ with $dist_G(u,w) = dist_G(v,w)$. \\
\linebreak 
\boldmath
\textbf{$(\Leftarrow)$ $G$ is \textit{only} bipartite if for any $uv \in E(G)$ there is no $w$ such that $dist_G(u,w) = dist_G(v,w)$.}  \\
\unboldmath
\linebreak 
It is sufficient to show that, as no such $w$ exists, the graph cannot contain any odd-length cycles, which is a necessary and sufficient condition for bipartite graphs. \\
\linebreak 
There are two ways in which a cycle of odd length can be created: 
\begin{enumerate}
    \item For any pair of adjacent vertices $u$ and $v$, there exists a third vertex $w$ s.t. $dist(u, w) = d(v, w)$. We have shown that no such vertex exists. 
    \item More generally, if there exists some vertex $w$ such that $uPw$ and $vPw$ both have even length. However, per the definition of a bipartite graph, this can never occur. WLOG, we can assume that $u, w\in V(A)$ and $v \in V(B)$. Any path $uPw$ must have even length, whereas any path $vPw$ must have odd length. Hence, it is not possible that both paths have an even length, and so we cannot create any odd-length cycles in this way either. 
\end{enumerate}
This shows that $G$ can have no odd cycles and is therefore bipartite. Only if some vertex $w$ would exist, we would be able to create an odd cycle. This shows that a connected graph $G$ is bipartite if and only if for any $uv \in E(G)$ there is no $w \in V(G)$ with $dist_G(u,w) = dist_G(v,w).$ \qed