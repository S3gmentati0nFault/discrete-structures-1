\subsection*{Exercise 5}
\boldmath
\textbf{Let $G$ be a graph on n vertices. Use BFS to show the following.}
\begin{enumerate}[a)]
    \item \textbf{Suppose that $G$ is $k$-regular graph and has girth $g(G) = 4$. Show that $n(G) \leq 2k$.} \\
    \linebreak 
    \unboldmath
    We know that the girth of a graph is the size of the smallest cycle and that $k$-regular means that $d(v) = |N(v)| = k$ $\forall v \in V(G)$. \\
    \linebreak 
    In BFS starting at a given vertex $v_1$ we are first exploring all vertices in $N(v_1)$ before moving on to the next vertex to explore (which will be $\in N(v_1)$). \\
    \linebreak 
    As $G$ has a girth of 4, there can be no cycles of length 3. This means that for any $v \in V(G)$, there can be no adjacent vertices in $N(v)$. This would create a cycle of length 3. \\
    \linebreak 
    Let $N(v_1) = \{k_1, ..., k_k\}$. We know that $\forall k_i \in N(v_1)$ $ N(k_i)$ includes $v_1$ but has $k-1$ vertices other than $v_1$ and that are not $\in N(v_1)$, i.e. $N(v_1) \cap N(k_i) = \emptyset$. There is nothing preventing overlap in the neighbourhoods of $k_1, ... k_k$, however. \\
    \linebreak 
    To the contrary, it holds that $N(N(v_1))$ is one set of size $k$, with $k-1$ new vertices, i.e. $N(k_1) = N(k_2) = ... = N(k_i)$. We know immediately that it holds that there must be some overlap of the neighbours in order to create a cycle of length 4. But in fact, all sets must be equal, otherwise it would not be possible to to build a $k$-regular graph. \\
    \linebreak 
    Finally, we know that $N(N(N(v_1)))$ cannot exist, as all vertices we have explored thus far already have a degree $k$. \\
    \linebreak 
    To conclude, in BFS we would explore vertex $v_1$ followed by the $k$ vertices in $\in N(v_1)$ followed by $k-1$ vertices in $N(N(v_1))$. Further vertices cannot exist in the graph, as this would violate $k$-regularity. So, we know that $G$ can have at most $1 + k + k-1 = k + k = 2k$ vertices. \qed
    \boldmath
    \item \textbf{Suppose that $\delta(G) \geq k$ and $g(G) = 5$. Show that $n(G) \geq k^2 + 1$.} \\
    \linebreak 
    \unboldmath
    As $\delta(G) \geq k$, all vertices must have a degree of at least $k$. We also know that as the girth is = 5, there can be no cycles of length 3 or 4. \\
    \linebreak 
    As mentioned in the previous question, we know that a cycle of length 3 is created if, for any $v \in V(G)$, a pair of vertices $\in N(v)$ are adjacent. Furthermore, a cycle of length 4 can be created in one of two ways:
    \begin{itemize}
        \item For any vertex $v$ and any three vertices $k_1, k_2, k_3 \in N(v)$, two pairs of vertices are adjacent. 
        \item For any vertex $v$ and any two vertices $k_1$ and $k_2 \in N(v)$, $k_1$ and $k_2$ share a neighbour \textbf{\boldmath $\neq v$ \unboldmath} 
    \end{itemize}
    From the second statement, we know that each vertex $\in N(v_1)$ linked to at least $k-1$ distinct vertices as they share the common neighbour $v_1$ but any further neighbours in common would result in a cycle of length 4. \\
    \linebreak
    When using BFS, we would first explore all vertices $\in N(v_1)$. After this vertex and its neighbours have been explored, we have explored $ \geq 1 + k$ vertices. In the next iteration, we explore the neighbourhood of all vertices $\in N(v_1)$. \\
    \linebreak 
    In this iteration, we have explored all vertices in $N(N(v_1))$. As each neighbour of $v_1$ has at least $k-1$ distinct neighbours, we have explored $\geq k(k-1)$ vertices. In total, therefore, we have explored $ \geq 1 + k + k(k-1) = 1 + k + k^2 - k = \geq 1 + k^2$. \\
    \linebreak 
    In further iterations, neighbourhoods do not need to consist of at least $k-1$ distinct vertices, as common vertices would result in cycles of length 5, which is allowed. Regardless, the requirement that $n(G) \geq k^2 + 1$ is already satisfied. \\
    \linebreak 
    In conclusion, for a graph $G$ with $\delta(G) \geq k$ and $g(G) \geq 5$, $n(G) \geq k^2 + 1$. \qed
\end{enumerate}
