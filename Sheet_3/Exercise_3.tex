\subsection*{Exercise 3}
\boldmath 
\textbf{Let $T$ be a tree with no vertex of degree $2$ and $n(T) \geq 2$. Show that $T$ has more leaves than
other vertices. \spacer
Proof} \\
\unboldmath
We will prove the hypothesis via induction over the subtrees in $T$. The idea is the following.\\ 
\linebreak 
We will call the set of the leaves of the tree $V_l(T)$ and the set of internal nodes in the tree as
$V_i(T)$.\\ 
\linebreak 
In the base case we just have the root with only one son, the
property trivially holds.\\
\linebreak
The more interesting case is the inductive one, in which we have a generic tree, for which we suppose that $|V_l(T)| > |V_i(T)|$ holds.\\
\linebreak 
If we add a subtree the my current tree, independently from the number of sons, we will just be turning one of the leaves in the root of the subtree and we will be attaching to it $k$ sons. This can be translated into the following
\begin{equation}
    |V_l(T)| + k > |V_i(T)| + 1
\end{equation}
Based on the structure of the subtree we have two different scenarios:
\begin{itemize}
    \item If the subtree is composed only by the root then we will have $k = 1$ and the number of internal vertices is unchanged. Thus the property trivially holds.
    \item If the subtree contains more than one son\footnote{We won't consider the subtree containing the root and just one son because the root of the tree would have degree $2$ and that is against the original hypothesis} the property holds because $k > 1$ and $|V_l(T)| > |V_i(T)|$ by induction hypothesis.
\end{itemize}
Thus in the end we have that $|V_l(T)| + k > |V_i(T)| + 1$, which means that the property still holds, thus
proving that the number of leaves is always greater than the number of internal vertices for any non-binary tree. \qed