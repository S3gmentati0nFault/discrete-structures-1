\subsection*{Exercise 4}
\boldmath
\textbf{Prove or disprove: If $e$ is an edge of a connected graph $G$, then $e$ belongs to some spanning tree of $G$.\spacer 
Proof} \\
\unboldmath
Let $G'$ be the result of removing $e$ from graph $G$. $G'$ can either be connected or disconnected. \\
\linebreak
\textbf{Case 1}: $G'$ is disconnected \\
If $G$ disconnects the graph, it means there is no longer a $uv$ path $\forall u,v \in V(G)$, which also means there cannot be a spanning tree of $G$ (as not all vertices are accounted for). This means that $e$ was a so-called \textit{bridge} edge, and by definition is contained in every spanning tree $T$ of $G$. As $e$ is contained in all spanning trees, it is contained in some spanning tree. \\
\linebreak 
\textbf{Case 2}: $G'$ is connected \\
As $G'$ is still connected, we know by definition there exists some spanning tree $T$ not containing the edge $e$. Since $T$ is a spanning tree, $V(G) = V(T)$, since a tree is minimally connected and maximally acyclic (by theorem 1.15) we cannot add the edge $e$ without closing a cycle $\cycl{C}$. \\
\linebreak 
Let $x$ be some edge in the cycle $\cycl{C}$ that is not $e$. We can remove $x$ from the cycle $\cycl{C}$ and be left with a path $Y$ that connects all vertices $u, v \in \cycl{C}$. Although we changed the structure of the spanning tree $T$ the connectivity still holds, thus we have built a different spanning tree $T'$, containing $e$ starting from $T$. \\
\linebreak
That proves the thesis. \qed
