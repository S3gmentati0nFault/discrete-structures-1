\subsection*{Exercise 4}
\boldmath
\textbf{Show that every tree $T$ has at least $\Delta(T)$ leaves.}\\\linebreak
\unboldmath
We can prove the statement by induction over the number of vertices $x$.\\\linebreak
In the base case $x = 1$ we know that the number of leaves is just $1$ and the maximum degree $\Delta(T) = 1$.\\\linebreak
Let's now suppose that the graph has number of vertices $x$ and the number of leaves is at least $\Delta(T)$, we want to prove that by adding a vertex the statement is still true.\\\linebreak
If we call $M$ the set of vertices of maximum degree in $T$, there are only three ways for us to add a vertex $v$ to the tree:
\begin{itemize}
       \item $v \in N_T(M)$\footnote{we define $N_T(M)$ where $M$ is a set as $\{x \in V(T) \st \exists y \in M, (x, y) \in E(T)\}$}. In which case $\Delta_{x + 1}(T) = \Delta_{x}(T) + 1$ but we know by inductive hypothesis that:
       \begin{equation*}
              L_x(T) \geq \Delta_x(T)
       \end{equation*}
       Where $L_x(T)$ is the set of leaves for the tree with $x$ vertices. Because of what we said above we know that:
       \begin{align*}
              &L_{x + 1}(T) \geq \Delta_{x + 1}(T) \\
              &L_x(T) + 1 \geq \Delta_x(T) + 1
       \end{align*}
       Which shows that the statement still holds.
       \item $v \notin N_T(M)$ and we do not modify the structure of the tree, thus we are just adding a leaf and that, by induction hypothesis, means that the statement still holds.
       \item We modify the structure of the tree, basically we are doing a subdivision of an edge, but that doesn't change the number of leaves in the tree, which in turn means that the statement still holds by induction hypothesis.
\end{itemize} 
Since independently from how we add a new vertex the statement will be true we just proved that the number of leaves in a tree will always be at least $\Delta(T)$.\qed