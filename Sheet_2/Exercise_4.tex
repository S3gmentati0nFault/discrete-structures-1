\section*{Exercise 4}
\boldmath
\textbf{Show that every tree $T$ has at least $\Delta(T)$ leaves.}\\
\unboldmath
\linebreak
Let's consider a generic tree $T$ with a vertex $v$ of maximum degree $\Delta T$, i.e. $d(v) = \Delta(T)$. \\
\linebreak 
We know that $\forall v \in V(T)$, $v$ is either a leaf vertex, or there exists a path to a leaf. We know this holds because of theorem 1.15:
\begin{itemize}
       \item There must be a (unique) path between any two nodes $u$ and $v$
       \item The tree $T$ is minimally connected
\end{itemize}
If any of the above doesn't hold for any reason then we can't say that $T$ is a tree. 
\\
\linebreak 
Therefore, we know that $\exists$ a path to as many leaves as there are neighbours $\in N_T(v)$. If that weren't true that would mean that one of the leaves in the tree is disconnected from $v$ and thus $T$ would not be a tree. We must finally show that each neighbour that is not a leaf has a path to a \textit{unique} leaf. \\
\linebreak 
We can prove this easily using a contradiction. Assume that for two vertices, $u, w \in N_T(v)$,
there exists a path to the same vertex leaf $x$. As $x$ is a leaf, it has degree 1.\\
\linebreak 
So, the two paths $uPx$ and $wPx$ exist. We also know the following path exists: $uPw$, as $u$ and $w$ are neighbours of $v$. The concatenation of these paths, $uPw \cup wPx \cup xPu$ forms a cycle, as it starts and ends at the same vertex, which cannot hold as trees are by definition acylic.  \\
\linebreak 
By this contradiction, we know that any vertex is linked to as many unique leaves as the size of its neighbourhood. Therefore, any tree $T$ has at least $|N_T(v)| = d(v) = \Delta(T)$ leaves. $\hspace{10mm} \square$ 
