\section*{Exercise 4}
\boldmath
\textbf{Show that every tree $T$ has at least $\Delta(T)$ leaves.}\\
\unboldmath
\linebreak
Let's consider a generic tree $T$ with a vertex $v$ of maximum degree $\Delta T$, i.e. $d(v) = \Delta(T)$. \\
\linebreak 
We know that $\forall v \in V(T)$, $v$ is either a leaf vertex, or there exists a path to a leaf. We know this holds because of theorem 1.15:
\begin{itemize}
       \item There must be a (unique) path between any two nodes $u$ and $v$
       \item The tree $T$ is minimally connected
\end{itemize}
If any of the above doesn't hold for any reason then we can't say that $T$ is a tree. 
\\
\linebreak 
Therefore, we know that $\exists$ a path to as many leaves as there are neighbours $\in N_T(v)$. If that weren't true that would mean that one of the leaves in the tree is disconnected from $v$ and thus $T$ would not be a tree. We must finally show that each neighbour that is not a leaf has a path to a \textit{unique} leaf. \\
\linebreak 
To prove that we can think about this we'll consider the set of vertices $N(v)$, if we consider the
set of vertices of each of the subtrees rooted in $x \in N(v)$, it's apparent that the intersection
between all of the different sets will be $\emptyset$.\\ 
If that weren't true we would have different subtrees containing the same nodes, meaning that for
two different roots $u, w \in N(v)$ we have a vertex $y$ such that, $uPy \cup yPw \cup wv \cup vu$
is a cycle, which violates the acyclicity of the tree.\\ 
\linebreak 
We can iterate this logic considering the subtrees built starting from the descendants of the
vertex, if there are no descendants it's clear that the vertex is linked to one single leaf.\\ 
\linebreak
By this contradiction, we know that any vertex is linked to as many unique leaves as the size of its neighbourhood. Therefore, any tree $T$ has at least $|N_T(v)| = d(v) = \Delta(T)$ leaves. $\hspace{10mm} \square$ 
