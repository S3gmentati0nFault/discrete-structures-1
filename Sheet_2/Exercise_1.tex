\section*{Exercise 1}
\boldmath
\textbf{Let $G$ be a graph.}
\begin{enumerate}[a)]
    \item \textbf{Prove or disprove: If $u$ and $v$ are the only vertices of odd degree in $G$, then $G$ contains a $u–v$ path.} 
    \unboldmath
    \\
    \linebreak 
    %It should be noted that a cycle is also a path, with the extra condition that for a path $uPv, u=v$ the case that $u = v$, this does not hold.
    %\boldmath
    To prove this we can use the handshaking lemma: "The number of vertices of odd degree in a graph is always even." (from \textit{Graph Theory} by Reinhardt Diestel, page 5)\\
    \linebreak 
    In the case that $G$ is connected, the statement holds trivially as by definition, $\exists$ a $uv$ path. Hence, we will take a disconnected graph $G$. \\
    \linebreak 
    As the graph is disconnected, $\exists $ at least 2 components of the graph that are not connected. We can use a proof by contradiction to show that $u$ and $v$ must be in the same component, and so $\exists$ a $uv$ path. \\
    \linebreak
    Take $G_1$ and $G_2$ to be two connected components of $G$, where $u \in V(G_1)$ and $v \in V(G_2)$. By the statement, there are no other vertices of odd degree, and so at present both $G_1$ and $G_2$ currently have exactly 1 vertex of odd degree each. \\
    \linebreak 
    As $G_1$ and $G_2$ are graphs in their own right, the handshaking lemma must hold. As $G_1$ and $G_2$ currently both have an odd number of odd degree vertices, the handshaking lemma does not hold, and so we have a contradiction. \\
    \linebreak 
    By this contradiction, $u$ and $v$ must both be in the same component, i.e. $u, v \in V(G_1)$ or $u, v \in V(G_2)$ and so by definition $\exists$ a $uv$ path. \hspace{10mm} $\square$ \\
    \linebreak 
    %By the handshaking lemma, this is not possible, as the 
    \boldmath
    \item \textbf{Let $uv \in E(G)$. Prove that $uv$ belongs to at least $d(u)+d(v)-n(G)$ triangles in $G$. (A triangle is a cycle of length 3.)} 
    \unboldmath
    \\
    \linebreak 
    We can prove the statement by induction on the number of vertices in $G$, i.e. $n(G)$. Let $t(G, uv)$ denote the number of triangles in $G$ that $uv$ belongs to. \\
    \linebreak 
    Case 1: $n(G) = 2$: $G$ contains only vertices $u$ and $v$ meaning $d(u) = d(v) = 1$. So, $t(G, uv) = 1 + 1 - 2 = 0$, which must hold as by definition there are too few vertices to form a cycle of length 3 (or any cycle for that matter). \\
    \linebreak 
    Case 2: $n(G) \geq 3$. Inductive hypothesis: for any graph with $n(G)$ nodes the number of triangles sharing the $uv$ edge is at least $d(u)+d(v)-n(G)$\\
    \linebreak
    In the generic case, if we suppose to add another vertex, we shall call it $w$ to $V(G)$ we can observe three different behaviours:
    \begin{itemize}
        \item $w$ is not linked to $u$ or $v$. If there is no path $P$ such that $wPx$ for any $x \in \{u, v\}$ then adding $w$ will just result in an increase in the number of total vertices\footnote{In some cases the number of vertices is greater than the sum of the degrees and thus the result of $d(u)+d(v)-n(G)$ is negative, but since it's a lower bound this poses no problem} which decreases the lower bound, and thus the property trivially holds even in the case $n(G) + 1$
        \item $w$ is linked to either $u$ or $v$ (not both). In this case we will have an increase in the amount of vertices inside the network, the increase of one of the degrees (either $d(v)$ or $d(u)$) will actually make it so that the lower bound remains the same. Thus the conclusion is the same as above.
        \item $w$ is linked to both $u$ and $v$. In this case the sum of the degrees will increase by two, but $n(G)$ increases by one, thus increasing $d(u)+d(v)-n(G)$ by just 1 (we are actually adding a triangle built on $uv$). Thus the property will still hold.
    \end{itemize}
    Since we can only fall in one of the three cases above we can say that if the property holds for $n(G)$ then it will still hold for any value greater than $n(G)$. We have thus proven by induction that the number of triangles in a graph sharing a certain edge $uv$ is at least $d(u)+d(v)-n(G)$ \hspace{10mm} $\square$ \\
\end{enumerate}