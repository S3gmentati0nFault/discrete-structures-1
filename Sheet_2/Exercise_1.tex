\section*{Sheet 2}
\subsection*{Exercise 1}
\boldmath
\textbf{Let $G$ be a graph.}
\begin{enumerate}[a)]
    \item \textbf{Prove or disprove: If $u$ and $v$ are the only vertices of odd degree in $G$, then $G$ contains a $u–v$ path.} 
    \unboldmath
    \\
    \linebreak 
    To prove this we can use the handshaking lemma: ‘The number of vertices of odd degree in a graph is always even.’ (from \textit{Graph Theory} by Reinhardt Diestel, page 5)\\
    \linebreak 
    In the case that $G$ is connected, the statement holds trivially as by definition, $\exists$ a $uv$ path. Hence, we will take a disconnected graph $G$. \\
    \linebreak 
    As the graph is disconnected, $\exists $ at least 2 components of the graph that are not connected. We can use a proof by contradiction to show that $u$ and $v$ must be in the same component, and so $\exists$ a $uv$ path. \\
    \linebreak
    Take $G_1$ and $G_2$ to be two connected components of $G$, where $u \in V(G_1)$ and $v \in V(G_2)$. By the statement, there are no other vertices of odd degree, and so at present both $G_1$ and $G_2$ currently have exactly 1 vertex of odd degree each. \\
    \linebreak 
    As $G_1$ and $G_2$ are graphs in their own right, the handshaking lemma must hold. As $G_1$ and $G_2$ currently both have an odd number of odd degree vertices, the handshaking lemma does not hold, and so we have a contradiction. \\
    \linebreak 
    By this contradiction, $u$ and $v$ must both be in the same component, i.e. $u, v \in V(G_1)$ or $u, v \in V(G_2)$ and so by definition $\exists$ a $uv$ path. Hence, for a graph containing vertices $u$ and $v$ as the only two vertices of odd degree, there must exist a $uv$ path. $\hspace{10mm} \square$ \\
    \linebreak 
    \boldmath
    \item \textbf{Let $uv \in E(G)$. Prove that $uv$ belongs to at least $d(u)+d(v)-n(G)$ triangles in $G$. (A triangle is a cycle of length 3.)} 
    \unboldmath
    \\
    \linebreak 
    We can prove the statement by induction on the number of vertices in $G$, i.e. $n(G)$. \\
    \linebreak 
    Case 1: $n(G) = 2$: $G$ contains only vertices $u$ and $v$ as well as edge $uv$, meaning $d(u) = d(v) = 1$. So, $d(u)+d(v)-n(G) = 1 + 1 - 2 = 0$, which must hold as by definition there are too few vertices to form a cycle of length 3 (or any cycle for that matter). \\
    \linebreak 
    Case 2: $n(G) \geq 3$. Inductive hypothesis: for any graph $G = (V,E)$ with $n(G)$ vertices and an edge $uv \in E(G)$, it holds that the number of triangles sharing the $uv$ edge is $ \geq d(u)+d(v)-n(G)$. \\
    \linebreak
    We must now show that the theorem still holds when any number of edges and vertices are added, i.e. the inductive hypothesis holds for any value of $n(G)$ and $e(G)$. \\
    \linebreak 
    When adding a vertex $w$ to the graph $G$ we can observe exactly one of three different behaviours:
    \begin{enumerate}
        \item $w$ is not adjacent to $u$ or $v$, i.e. $w \notin N_G(u) \wedge w \notin N_G(v)$. This means $d(u)$ and $d(v)$ remain the same while $|V(G)|$ increases by $1$, thus decreasing $d(u)+d(v)-n(G)$ by $1$. This means the statement will still hold, as it is impossible for new triangles to have been formed which include the edge $uv$. Note that it is possible that the number of vertices is greater than $d(u)+d(v)$ and thus the result of $d(u)+d(v)-n(G)$ is negative, but since this is a lower bound this poses no problem. 
        \item $w$ is adjacent to either $u$ or $v$, i.e. $w \in N_G(u) \vee w \in N_G(v)$. This means $|V(G)|$ increases by $1$ and either $d(v)$ or $d(u)$ is increased by one. This means $ d(u)+d(v)-n(G)$ is not changed, which holds true because, as in the case above, it is not possible for any triangles including edge $uv$ to have been added. 
        \item $w$ is adjacent to both $v$ or $w$, i.e. $w \in N_G(u) \wedge w \in N_G(v)$. This means both $d(u)$, $d(v)$ and $n(G)$ increased by $1$, thus $d(u)+d(v)-n(G)$ increases overall by $1$. Hence, a new triangle has been added, which makes sense given we now have edges $uw$, $wv$ and $vu$ and so we have a (new) $u-w-v-u$ path of exactly length 3, which is the definition of triangle.  
    \end{enumerate}
    In conclusion, in any of the three possible cases, the theorem holds, meaning that for a graph $G = (V,E)$ and an edge $uv \in E(G)$, it holds that $uv$ belongs to at least $d(u)+d(v)-n(G)$ triangles. \hspace{10mm}$\square$
\end{enumerate}