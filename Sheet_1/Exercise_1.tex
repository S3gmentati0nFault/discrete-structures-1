\section*{Sheet 1}
\subsection*{Exercise 1}
\boldmath
\textbf{A graph is \textit{connected} if, for any vertices $u, v \in V(G), G$ contains a $u - v$ path.\\
Prove that a graph or its complement is connected, possibly both.\\
\linebreak
Proof\\}
\unboldmath
Let's suppose, for a contradiction, that neither $G$ nor $\compl{G}$ are connected.\\
\linebreak
We can suppose, without loss of generality, that $G$ is composed by two or more connected components, we will call the connected components $G_1, \dots, G_k$. Since $G$ is not connected the edges linking the various connected components are missing.\\
Due to how the complementation operation carries out, inside $\compl{G}$ all of the vertices from $G_1$ will be connected to all of the vertices from $G_2, \dots, G_k$, if that weren't true $G$ would already be connected and that would contradict our hypothesis.\\
What I showed for component $G_1$ can be extended to all the other $k$ components.\\
\linebreak
Since all of the vertices in the $i$-th component are connected to all of the other vertices in the graph apart from the ones inside the $i$-th component, $\compl{G}$ is trivially connected and we reached an absurd.\\
\linebreak
The same can be concluded if $\compl{G}$ is not connected. If we consider the case in which either $G$ or $\compl{G}$ are connected, as long as the starting graph is missing, at least, $|V(G)|$ edges in $E(G)$ then its complement will be connected as well\footnote{If we suppose $|V(G)| = n$ and we suppose that the previous statement weren't true and that $G$ was missing exactly $n$ edges to make it complete. If we take the subgraph $G'$ formed just by the missing edges and the vertices insisting on those edges, it would mean that $G'$ would not be connected, and that is only possible if the amount of vertices in $G$ is greater than $n$. Because we would need at least an edge linking two vertices out of reach for the main component, thus we reached an absurd because the amount of vertices in $G$ is exactly $n$.}.\qed