\subsection*{Exercise 2}
\textbf{Use complete graphs and counting arguments to show the following.}
\begin{enumerate}[\textbf{a)}]
    \boldmath
    \item $\binom{n}{2} = \binom{k}{2} + k(n-k) + \binom{n-k}{2} \text{ \textbf{for} } 0 \leq k \leq n $ \\
    \linebreak
    \unboldmath 
    Let's suppose we are dealing with a clique $G$ of size $n$, $\binom{n}{2}$ is the number of all the possible subsets of size $2$ that we can build in $G$, thus $E(G)$.\\\linebreak
    Given $k \in [0, n]$ we want to count the amount of edges in $G$ in a different way, the equation built in the hypothesis uses three different components:
    \begin{itemize}
        \item $\binom{k}{2}$ is the size of the set of all the possible edges inside a subset $X \subseteq V(G)$ of size $k$.
        \item $k(n - k)$ is the number of the edges connecting vertices in $X$ and vertices in $V(G - X)$
        \item $\binom{n - k}{2}$ the size of the set of all possible connections between any two vertices in $G - X$
    \end{itemize}
    Therefore the total number of edges in any clique can be computed as follows:
    \begin{equation*}
        \binom{n}{2} = \binom{k}{2} + k(n-k) + \binom{n-k}{2}
    \end{equation*}
    \qed
    \boldmath
    \item \textbf{If $\Sigma^k_{i=1}n_i = n \text{ for } n_1, ..., n_k \in \mathbb{N}_0, \text{ then } \Sigma^k_{i=1} \binom{n_i}{2} \leq \binom{n}{2}$} \\\linebreak 
    \unboldmath
    If I take any partition of the vertex set of a clique and I compute a sum over the amount of vertices in each subclique $\binom{n_i}{2}$ the result will always be smaller thant the amount of edges in the original clique. That is because by computing $\sum_{i = 1}^k \binom{n_i}{2}$ we are only counting the edges inside the subclique but not the edges joining the various subcliquest together, thus
    \begin{equation*}
        \sum_{i = 1}^k \binom{n_i}{2} \leq \binom{n}{2}
    \end{equation*}
    \qed
\end{enumerate}