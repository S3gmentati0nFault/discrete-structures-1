\subsection*{Exercise 3}
\boldmath
\textbf{Show that any graph $G$ contains a subgraph of minimun degree at least $\frac{d(G)}{2}$}\vspace{10pt}\\
\unboldmath
A subgraph of $G$ is, by definition, a graph built on top of a subset of the edges of $G$ and a subset of the vertices of $G$.\\
I am not really interested in the structure of the graph I am starting to build upon, if the graph is composed just by vertices and there are no edges I think that picking one random vertex in the mix should do the trick and grant the property.\\
In a broader an more generic scenario I will provide a proof by construction, the idea is the following:\\
We start building a generic $G'$ graph which is a pair built this way:
\begin{enumerate}
    \item $V(G')$ - set of vertices that have nonzero degree.
    \item $E(G')$ - $\emptyset$
\end{enumerate}
Now I will pick the vertex of minimum degree in $G$\footnote{\textbf{Side note}: if $\delta(G) < \floor{\frac{d(G)}{2}}$ I will simply ignore the vertex and pick another one as the "minimum degree node" for $G'$ so that I have enough edges to validate the minimum degree statement. The rest of the construction process stays the same.} and call it $w$ in $G'$, I then add $\floor{\frac{d(G)}{2}}$ edges taken from $G$ to $G'$, the idea is to make sure that $d(w) = \floor{\frac{d(G)}{2}}$.\\
Now, to make sure that $d(w) = \delta(G')$ we want to add enough edges so that each vertex $v \neq w$ has $d(v) \geq \floor{\frac{d(G)}{2}}$, what's important for this constructive step is to make sure that vertex $w$ is left untouched.\\
At the end of the second construction step we will have built a subgraph for $G$ with minimum degree $\delta(G') = \floor{\frac{d(G)}{2}}$.
\vspace{2pt}\\\hspace*{2.5cm}$\square$