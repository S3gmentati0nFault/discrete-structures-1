\section*{Sheet 5}
\subsection*{Exercise 1}
\boldmath
\textbf{Let $G$ be a graph. A \textit{cycle decomposition} of $G$ is a partition of $E(G)$ into cycles (i.e., a set of edge-disjoint cycles of $G$ covering all the edges of $G$). Show that $G$ has a cycle decomposition if and only if each of its connected components contains an Euler tour}. \\
\unboldmath
\linebreak 
An Euler tour is cycle that uses each edge in the graph exactly once. It is a necessary and sufficient condition that a graph contains an Euler tour if all vertices have even degree. \\
\linebreak 
Therefore, we can show that $G$ has a cycle decomposition if and only if all vertices have even degree. WLOG we can consider each component separately. Let $F$ be a connected component of $G$. \\
\linebreak 
\boldmath
($\Rightarrow$) \unboldmath If $F$ has a cycle decomposition $\Rightarrow$ all vertices $\in F$ have even degree. \\
\linebreak 
As $F$ is connected, there are no isolated vertices (vertices of degree 0). Therefore, $\forall v \in V(F)$, $d_F(v) \geq 1$, meaning it belongs to $\geq 1$ cycles. In order for $v$ to belong to a cycle, it must have exactly 1 incoming and 1 outgoing edge. More generally, it will have 1 incoming and 1 outgoing edge for each cycle that it belongs to. This will always be an even number, and so the degree of each vertex is even. \\
\linebreak
\boldmath
($\Leftarrow$) \unboldmath If all vertices in $F$ have even degree $\Rightarrow$ $F$ has a cycle decomposition. We can prove this using induction on $e(F)$. \\
\linebreak 
\textbf{Base case:} In the general case, the base case would be $e(F) = 0$ (as there are no edges, all vertices have degree 0, which is even, and so the graph has a cycle decomposition, which is an empty set). As $F$ is connected, the base case will never occur and we can skip to the inductive step. \\
\linebreak 
\textbf{Inductive hypothesis}: Suppose that $F$ has $< e(F)$ edges. There will exist a cycle decomposition. We can use this to shown that any graph $F$ with $e(F)$ edges has a cycle decomposition. \\
\linebreak 
As all vertices have even degree, there exists at least 1 cycle $C$. We can remove the edges of this cycle from $F$, and so our remaining graph is $F \setminus E(C)$. $F \setminus E(C)$ will have the same vertex set, and since we are removing an even number of edges, the resulting graph will also have only even degree vertices. \\
\linebreak 
We can continue to remove the cycles, each time adding $C$ to a set $\mathcal{C}$. Eventually we will have a graph on $n(F)$ vertices and 0 edges and we have shown that the graph has cycle decomposition $\mathcal{C}$.
\\
\linebreak 
We have shown that the statement holds for any connected component, and will therefore hold for all connected components. $G$ has a cycle decomposition if and only if each of its connected components contains an Euler tour. \qed